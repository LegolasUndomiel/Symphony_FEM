\chapter{张量导数}

\section{随体坐标系}
\subsection{张量求导}
\begin{equation}
    \begin{split}
        \frac{\mathrm{d}\boldsymbol{\sigma}}{\mathrm{d}t}&=\frac{\mathrm{d}}{\mathrm{d}t}(\sigma_{ij}\vec{e}_i\vec{e}_j)\\
        &=\frac{\mathrm{d}\sigma_{ij}}{\mathrm{d}t}\vec{e}_i\vec{e}_j+\sigma_{mj}\frac{\mathrm{d}\vec{e}_m}{\mathrm{d}t}\vec{e}_j+\sigma_{in}\vec{e}_i\frac{\mathrm{d}\vec{e}_n}{\mathrm{d}t}\\
        &=\frac{\mathrm{d}\sigma_{ij}}{\mathrm{d}t}\vec{e}_i\vec{e}_j+\sigma_{mj}\qty(\epsilon_{mip}\Omega_p\vec{e}_i)\vec{e}_j+\sigma_{in}\vec{e}_i\qty(\epsilon_{njq}\Omega_q\vec{e}_j)\\
        &=\qty(\frac{\mathrm{d}\sigma_{ij}}{\mathrm{d}t}+\epsilon_{mip}\Omega_p\sigma_{mj}+\epsilon_{njq}\Omega_q\sigma_{ni})\vec{e}_i\vec{e}_j\\
        &=\qty(\frac{\mathrm{d}\sigma_{ij}}{\mathrm{d}t}+\epsilon_{ipm}\Omega_p\sigma_{mj}+\epsilon_{jqn}\Omega_q\sigma_{ni})\vec{e}_i\vec{e}_j
    \end{split}
\end{equation}

\begin{align}
    &\epsilon_{ipm}\Omega_p\sigma_{mj}=(\vec{\Omega}\times\boldsymbol{\sigma})_{ij}=(\textbf{W}\cdot\boldsymbol{\sigma})_{ij}\\
    &\epsilon_{jqn}\Omega_q\sigma_{ni}=(\vec{\Omega}\times\boldsymbol{\sigma})_{ji}=(\vec{\Omega}\times\boldsymbol{\sigma})^T_{ij}=(\boldsymbol{\sigma}\cdot\textbf{W}^T)_{ij}
\end{align}

\begin{equation}
    \dot{\boldsymbol{\sigma}}=\boldsymbol{\sigma}^{\nabla J}+\textbf{W}\cdot\boldsymbol{\sigma}+\boldsymbol{\sigma}\cdot\textbf{W}^{T}
\end{equation}