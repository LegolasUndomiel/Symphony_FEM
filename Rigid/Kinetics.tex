\chapter{刚体动力学}
\begin{introduction}
\item 角动量守恒定律
\item 动力学方程
\item 刚体数值算法
\item 刚体动能
\end{introduction}

\section{角动量守恒定律}
\subsection{固定点O}
\subsubsection*{质点}
\begin{equation}
  \sum \bm{F}=m\dot{\bm{v}}
\end{equation}

\begin{equation}
  \sum \bm{M}_O=\bm{r}\times\sum \bm{F}=\bm{r}\times m\dot{\bm{v}}
\end{equation}

\begin{equation}
  \dot{\bm{H}}_O=\frac{\mathrm{d}}{\mathrm{d}t}(\bm{r}\times m\bm{v})=\cancel{\dot{\bm{r}}\times m\bm{v}}+\bm{r}\times m\dot{\bm{v}}=\bm{r}\times\sum \bm{F}=\sum \bm{M}_O
\end{equation}
\begin{theorem}[固定点O质点角动量守恒定律]
  \begin{equation}
    \frac{\mathrm{d}\bm{H}_O}{\mathrm{d}t}=\sum \bm{M}_O
  \end{equation}
\end{theorem}

\subsubsection*{质点系与刚体}
\begin{equation}
  (\dot{\bm{H}}_i)_O=\bm{r}_i\times\bm{F}_i
\end{equation}
\begin{equation}
  \dot{\bm{H}}_O=\sum(\dot{\bm{H}}_i)_O=\sum\bm{r}_i\times\bm{F}_i=\sum\bm{M}_O
\end{equation}
\begin{theorem}[固定点O质点系与刚体角动量守恒定律]
  \begin{equation}
    \frac{\mathrm{d}\bm{H}_O}{\mathrm{d}t}=\sum \bm{M}_O
  \end{equation}
\end{theorem}

\subsection{质心G}
\begin{equation}
  (\bm{H}_i)_G=\bm{r}_{i/G}\times m_i\bm{v}_i
\end{equation}
\begin{equation}
  \begin{split}
    (\dot{\bm{H}}_i)_G&=\dot{\bm{r}}_{i/G}\times m_i\bm{v}_i+\bm{r}_{i/G}\times m_i\dot{\bm{v}}_i\\
    &=\bm{v}_{i/G}\times m_i\bm{v}_i+\bm{r}_{i/G}\times m_i\bm{a}_i\\
    &=\bm{v}_{i/G}\times m_i(\bm{v}_{i/G}+\bm{v}_G)+\bm{r}_{i/G}\times m_i\bm{a}_i\\
    &=(\bm{\omega}\times\bm{r}_{i/G})\times m_i\bm{v}_G+\bm{r}_{i/G}\times m_i\bm{a}_i
  \end{split}
\end{equation}

\begin{equation}
  \begin{split}
    \dot{\bm{H}}_G&=\sum(\dot{\bm{H}}_i)_G\\
    &=\sum\qty((\bm{\omega}\times\bm{r}_{i/G})\times m_i\bm{v}_G)+\sum\qty(\bm{r}_{i/G}\times m_i\bm{a}_i)\\
    &=\qty(\bm{\omega}\times\qty(\bcancel{\sum m_i\bm{r}_{i/G}}))\times\bm{v}_G+\sum\bm{M}_G\\
    &=\sum\bm{M}_G
  \end{split}
\end{equation}

\begin{theorem}[质心G角动量守恒定律]
  \begin{equation}
    \frac{\mathrm{d}\bm{H}_G}{\mathrm{d}t}=\sum \bm{M}_G
  \end{equation}
\end{theorem}

\subsection{任意点A}
单个质点:
\begin{equation}
  (\bm{H}_i)_A=\bm{r}_{i/A}\times m_i\bm{v}_i
\end{equation}

对时间求导:
\begin{equation}
  \begin{split}
    (\dot{\bm{H}}_i)_A&=\dot{\bm{r}}_{i/A}\times m_i\bm{v}_i+\bm{r}_{i/A}\times m_i\dot{\bm{v}}_i\\
    &=\bm{v}_{i/A}\times m_i\bm{v}_i+\bm{r}_{i/A}\times m_i\bm{a}_i\\
    &=\bm{v}_{i/A}\times m_i(\bm{v}_{i/A}+\bm{v}_A)+\bm{r}_{i/A}\times m_i\bm{a}_i\\
    &=(\bm{\omega}\times\bm{r}_{i/A})\times m_i\bm{v}_A+\bm{r}_{i/A}\times m_i\bm{a}_i
  \end{split}
\end{equation}

刚体:
\begin{equation}
  \begin{split}
    \dot{\bm{H}}_A&=\sum(\dot{\bm{H}}_i)_A\\
    &=\sum\qty((\bm{\omega}\times\bm{r}_{i/A})\times m_i\bm{v}_A)+\sum\qty(\bm{r}_{i/A}\times m_i\bm{a}_i)\\
    &=\qty(\bm{\omega}\times\qty(\sum m_i\bm{r}_{i/A}))\times\bm{v}_A+\sum\bm{M}_A\\
    &=\boxed{(\bm{\omega}\times\bm{r}_{G/A})\times m\bm{v}_A+\sum\bm{M}_A}\\
    &=(\bm{\omega}\times\bm{r}_{G/A})\times m(\bm{v}_G+\bm{\omega}\times\bm{r}_{A/G})+\sum\bm{M}_A\\
    &=\boxed{(\bm{\omega}\times\bm{r}_{G/A})\times m\bm{v}_G+\sum\bm{M}_A}\\
    &=\bm{v}_{G/A}\times m\bm{v}_A+\sum\bm{M}_A\\
    &=(\bm{v}_G-\bm{v}_A)\times m\bm{v}_A+\sum\bm{M}_A\\
    &=\boxed{\bm{v}_A\times(-m\bm{v}_G)+\sum\bm{M}_A}
  \end{split}
\end{equation}
\begin{note}
  这里的$\bm{r}_{G/A}$是任意点A到质心G的矢量,$\bm{v}_{G/A}$是质心G相对任意点A的速度,$\bm{v}_G$是质心G的速度,$\bm{v}_A$是任意点A的速度。
\end{note}

总结起来,任意点A处的角动量守恒定律可以写为以下三种形式:
\begin{proposition}[任意点A角动量守恒定律]\label{prop:rigid momentum A}
  \begin{align}
    \frac{\mathrm{d}\bm{H}_A}{\mathrm{d}t}&=(\bm{\omega}\times\bm{r}_{G/A})\times m\bm{v}_A+\sum\bm{M}_A\\
    \frac{\mathrm{d}\bm{H}_A}{\mathrm{d}t}&=(\bm{\omega}\times\bm{r}_{G/A})\times m\bm{v}_G+\sum\bm{M}_A\\
    \frac{\mathrm{d}\bm{H}_A}{\mathrm{d}t}&=\boxed{\bm{v}_A\times(-m\bm{v}_G)}+\sum\bm{M}_A\label{eq:rigid momentum A}
  \end{align}
\end{proposition}

任意点A处的角动量和质心G处的角动量有以下关系:
\begin{equation}
  \bm{H}_A=\bm{r}_{G/A}\times m\bm{v}_G+\bm{H}_G
\end{equation}
\begin{equation}
  \begin{split}
    \dot{\bm{H}_A}&=\dot{\bm{r}}_{G/A}\times m\bm{v}_G+\bm{r}_{G/A}\times m\dot{\bm{v}}_G+\dot{\bm{H}}_G\\
    &=(\bm{\omega}\times\bm{r}_{G/A})\times m\bm{v}_G+\bm{r}_{G/A}\times m\bm{a}_G+\dot{\bm{H}}_G
  \end{split}
\end{equation}

带入角动量守恒定律整理可以得到:
\begin{proposition}[力矩形式的达朗贝尔原理]
  \begin{equation}
    \frac{\mathrm{d}\bm{H}_G}{\mathrm{d}t}=\sum\bm{M}_A+\bm{r}_{G/A}\times(-m\bm{a}_G)
    \label{eq:rigid momentum d'Alembert}
  \end{equation}
\end{proposition}

根据达朗贝尔原理,质心处有惯性力。
方程右边两项分别代表作用在刚体上的所有力对任意点A的力矩作用、惯性力对任意点A的力矩作用,所以方程右边是作用在刚体上的合力矩。
由于力矩可以任意移动而不改变刚体的运动状态,所以可以将任意点A移动到质心G处,然后方程左边是质心G处的角动量对时间的导数,右边是作用在质心G处的合力矩。
这个方程可以看作是力矩形式的达朗贝尔原理。

在有限元仿真中,这个任意点A通常是刚体的参考点,有限元中对刚体的约束通常是对参考点的约束。
如果对刚体的参考点施加了位移约束,那么刚体的受力情况还要考虑参考点上的约束反力,而这个反力对参考点A的力矩作用“恰好”是零,
这就是推导任意点A处角动量守恒定律的原因。
可以看出,在中心差分法框架下,任意点A角动量守恒定律\ref{prop:rigid momentum A}解耦了刚体的平动和转动,可以独立求解刚体的转动。
至于刚体的平动,可以利用质心平动方程、质心加速度和参考点A的加速度之间的关系以及参考点A的约束方程来求解。

\begin{note}
  $\boxed{\bm{v}_A\times(-m\bm{v}_G)}$这一项代表什么含义?
  对比\ref{eq:rigid momentum A}和\ref{eq:rigid momentum d'Alembert},
  方程\ref{eq:rigid momentum A}为什么无法像方程\ref{eq:rigid momentum d'Alembert}一样简洁优雅地体现出达朗贝尔原理?
  当任意点A约束所有位移时,这一项是零,方程变成固定点O角动量守恒定律,可知此时方程成立。
  当任意点A是刚体质心G时,这一项也是零,方程变成质心G角动量守恒定律,可知此时方程成立。
  当任意点A有位移自由度且不是刚体质心G时,刚体的平动和转动耦合,刚体的平动会影响$\bm{H}_A$和$\bm{M}_A$,这一项体现了刚体的平动对任意点A角动量守恒定律的影响。
\end{note}

\section{动力学方程}
\subsection{欧拉-拉格朗日方程}
\subsubsection*{固定点O}
\begin{equation}
  \frac{\mathrm{d}\bm{H}_O}{\mathrm{d}t}=\bm{M}_O
\end{equation}

\begin{equation}
  \bm{H}_O=[\bm{I}_O]\bm{\omega}
\end{equation}

\begin{definition}[固定点欧拉-拉格朗日方程]
  \begin{equation}
    [\bm{I}_O]\dot{\bm{\omega}}+\bm{\omega}\times[\bm{I}_O]\bm{\omega}=\bm{M}_O
  \end{equation}
\end{definition}

\subsubsection*{质心G}
\begin{equation}
  \frac{\mathrm{d}\bm{H}_G}{\mathrm{d}t}=\bm{M}_G
\end{equation}

\begin{equation}
  \bm{H}_G=[\bm{I}_G]\bm{\omega}
\end{equation}

\begin{definition}[质心欧拉-拉格朗日方程]
  \begin{equation}
    [\bm{I}_G]\dot{\bm{\omega}}+\bm{\omega}\times[\bm{I}_G]\bm{\omega}=\bm{M}_G
  \end{equation}
\end{definition}

\subsubsection*{任意点A}
\begin{equation}
  \frac{\mathrm{d}\bm{H}_A}{\mathrm{d}t}=\bm{v}_A\times(-m\bm{v}_G)+\bm{M}_A
\end{equation}
\begin{equation}
  \bm{H}_A=[\bm{I}_A]\bm{\omega}
\end{equation}

\begin{definition}[任意点A欧拉-拉格朗日方程]
  \begin{equation}
    [\bm{I}_A]\dot{\bm{\omega}}+\bm{\omega}\times[\bm{I}_A]\bm{\omega}=\bm{v}_A\times(-m\bm{v}_G)+\bm{M}_A
  \end{equation}
\end{definition}

\subsection{矩阵形式}
\subsection{约束}

\section{数值算法}
\subsection{Runge-Kutta方法}
\subsection{自由运动}
\subsection{约束一个自由度}
\subsubsection*{x方向}
\subsubsection*{y方向}
\subsubsection*{z方向}
\subsection{约束两个自由度}
\subsubsection*{xy方向}
\subsubsection*{yz方向}
\subsubsection*{zx方向}
\subsection{定点运动}

\section{动能}

\subsection{固定点O}
\subsection{质心}
\subsection{任意点}
