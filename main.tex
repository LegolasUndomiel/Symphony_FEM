\documentclass[lang=cn,newtx,10pt,scheme=chinese]{elegantbook}

\usepackage{amsmath}
\usepackage{cancel}
\usepackage{physics}

\title{Symphony of Finite Element Methods}
% \subtitle{学习笔记}

\author{王松}
% \institute{Elegant\LaTeX{} Program}
\date{\today}
\version{0.1}
% \bioinfo{自定义}{信息}

\setcounter{tocdepth}{3}

% \logo{logo-blue.png}
\cover{cover.jpg}

% 本文档命令
\usepackage{array}
\newcommand{\ccr}[1]{\makecell{{\color{#1}\rule{1cm}{1cm}}}}

% 修改标题页的橙色带
\definecolor{customcolor}{RGB}{32,178,170}
\colorlet{coverlinecolor}{customcolor}
\usepackage{cprotect}

\begin{document}

\maketitle
\frontmatter

\tableofcontents

\mainmatter

\part{数学基础}
\chapter{张量导数}

\section{随体坐标系}
\subsection{张量求导}
\begin{equation}
    \begin{split}
        \frac{\mathrm{d}\boldsymbol{\sigma}}{\mathrm{d}t}&=\frac{\mathrm{d}}{\mathrm{d}t}(\sigma_{ij}\vec{e}_i\vec{e}_j)\\
        &=\frac{\mathrm{d}\sigma_{ij}}{\mathrm{d}t}\vec{e}_i\vec{e}_j+\sigma_{mj}\frac{\mathrm{d}\vec{e}_m}{\mathrm{d}t}\vec{e}_j+\sigma_{in}\vec{e}_i\frac{\mathrm{d}\vec{e}_n}{\mathrm{d}t}\\
        &=\frac{\mathrm{d}\sigma_{ij}}{\mathrm{d}t}\vec{e}_i\vec{e}_j+\sigma_{mj}\qty(\epsilon_{mip}\Omega_p\vec{e}_i)\vec{e}_j+\sigma_{in}\vec{e}_i\qty(\epsilon_{njq}\Omega_q\vec{e}_j)\\
        &=\qty(\frac{\mathrm{d}\sigma_{ij}}{\mathrm{d}t}+\epsilon_{mip}\Omega_p\sigma_{mj}+\epsilon_{njq}\Omega_q\sigma_{ni})\vec{e}_i\vec{e}_j\\
        &=\qty(\frac{\mathrm{d}\sigma_{ij}}{\mathrm{d}t}+\epsilon_{ipm}\Omega_p\sigma_{mj}+\epsilon_{jqn}\Omega_q\sigma_{ni})\vec{e}_i\vec{e}_j
    \end{split}
\end{equation}

\begin{align}
    &\epsilon_{ipm}\Omega_p\sigma_{mj}=(\vec{\Omega}\times\boldsymbol{\sigma})_{ij}=(\textbf{W}\cdot\boldsymbol{\sigma})_{ij}\\
    &\epsilon_{jqn}\Omega_q\sigma_{ni}=(\vec{\Omega}\times\boldsymbol{\sigma})_{ji}=(\vec{\Omega}\times\boldsymbol{\sigma})^T_{ij}=(\boldsymbol{\sigma}\cdot\textbf{W}^T)_{ij}
\end{align}

\begin{equation}
    \dot{\boldsymbol{\sigma}}=\boldsymbol{\sigma}^{\nabla J}+\textbf{W}\cdot\boldsymbol{\sigma}+\boldsymbol{\sigma}\cdot\textbf{W}^{T}
\end{equation} % 张量导数

\part{连续介质力学}

\part{刚体}
\chapter{刚体运动学}
\section{数学基础}
\subsection{随体坐标系基矢量求导}
\subsection{随体坐标系矢量求导}
\section{速度分解}
\section{加速度分解}
\section{角动量}
\subsection{任意点的角动量}
\subsection{固定点的角动量}
\subsection{质心的角动量}
\subsection{相互关系}
\section{Inertia Tensor}
 % 刚体运动学
\chapter{刚体动力学}
\section{角动量守恒定律}
\section{动力学方程}
\section{数值算法}
\section{动能}
 % 刚体动力学

\part{有限元单元理论}
\chapter{单元理论} % 单元理论
\chapter{杆单元} % 杆单元
\chapter{桁架单元} % 桁架单元
\chapter{梁单元} % 梁单元
\chapter{膜单元} % 薄膜单元
\chapter{板单元} % 板单元
\chapter{壳单元} % 壳单元
\chapter{四面体单元} % 四面体单元
\chapter{六面体单元} % 六面体单元
\chapter{实体壳单元} % 实体壳单元
\chapter{弹簧单元} % 弹簧单元
\chapter{质量单元} % 质量单元
\chapter{阻尼单元} % 阻尼单元

\end{document}
