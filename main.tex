\documentclass[lang=cn,newtx,10pt,scheme=chinese]{elegantbook}

\usepackage{amsmath}
\usepackage{cancel}
\usepackage{physics}

\title{Symphony of Finite Element Methods}
% \subtitle{学习笔记}

\author{王松}
% \institute{Elegant\LaTeX{} Program}
\date{\today}
\version{0.1}
% \bioinfo{自定义}{信息}

\setcounter{tocdepth}{3}

% \logo{logo-blue.png}
\cover{cover.jpg}

% 本文档命令
\usepackage{array}
\newcommand{\ccr}[1]{\makecell{{\color{#1}\rule{1cm}{1cm}}}}

% 修改标题页的橙色带
\definecolor{customcolor}{RGB}{32,178,170}
\colorlet{coverlinecolor}{customcolor}
\usepackage{cprotect}

\begin{document}

\maketitle
\frontmatter

\tableofcontents

\mainmatter

\part{数学基础}
\chapter{张量导数}

\section{随体坐标系}
\subsection{张量求导}
\begin{equation}
    \begin{split}
        \frac{\mathrm{d}\boldsymbol{\sigma}}{\mathrm{d}t}&=\frac{\mathrm{d}}{\mathrm{d}t}(\sigma_{ij}\vec{e}_i\vec{e}_j)\\
        &=\frac{\mathrm{d}\sigma_{ij}}{\mathrm{d}t}\vec{e}_i\vec{e}_j+\sigma_{mj}\frac{\mathrm{d}\vec{e}_m}{\mathrm{d}t}\vec{e}_j+\sigma_{in}\vec{e}_i\frac{\mathrm{d}\vec{e}_n}{\mathrm{d}t}\\
        &=\frac{\mathrm{d}\sigma_{ij}}{\mathrm{d}t}\vec{e}_i\vec{e}_j+\sigma_{mj}\qty(\epsilon_{mip}\Omega_p\vec{e}_i)\vec{e}_j+\sigma_{in}\vec{e}_i\qty(\epsilon_{njq}\Omega_q\vec{e}_j)\\
        &=\qty(\frac{\mathrm{d}\sigma_{ij}}{\mathrm{d}t}+\epsilon_{mip}\Omega_p\sigma_{mj}+\epsilon_{njq}\Omega_q\sigma_{ni})\vec{e}_i\vec{e}_j\\
        &=\qty(\frac{\mathrm{d}\sigma_{ij}}{\mathrm{d}t}+\epsilon_{ipm}\Omega_p\sigma_{mj}+\epsilon_{jqn}\Omega_q\sigma_{ni})\vec{e}_i\vec{e}_j
    \end{split}
\end{equation}

\begin{align}
    &\epsilon_{ipm}\Omega_p\sigma_{mj}=(\vec{\Omega}\times\boldsymbol{\sigma})_{ij}=(\textbf{W}\cdot\boldsymbol{\sigma})_{ij}\\
    &\epsilon_{jqn}\Omega_q\sigma_{ni}=(\vec{\Omega}\times\boldsymbol{\sigma})_{ji}=(\vec{\Omega}\times\boldsymbol{\sigma})^T_{ij}=(\boldsymbol{\sigma}\cdot\textbf{W}^T)_{ij}
\end{align}

\begin{equation}
    \dot{\boldsymbol{\sigma}}=\boldsymbol{\sigma}^{\nabla J}+\textbf{W}\cdot\boldsymbol{\sigma}+\boldsymbol{\sigma}\cdot\textbf{W}^{T}
\end{equation} % 张量导数

\part{连续介质力学}

\part{刚体}
\chapter{刚体运动学}

\section{数学基础}
\subsection{随体坐标系基矢量求导}
\begin{equation}
  \frac{\mathrm{d}\vec{e}_i}{\mathrm{d}t}=\epsilon_{ijk}\vec{e}_j\omega_k
\end{equation}
\subsection{随体坐标系矢量求导}
\begin{equation}
  \begin{split}
    \frac{\mathrm{d}\vec{v}}{\mathrm{d}t}&=\frac{\mathrm{d}}{\mathrm{d}t}(v_i\vec{e}_i)=\frac{\mathrm{d}v_i}{\mathrm{d}t}\vec{e}_i+v_i\frac{\mathrm{d}\vec{e}_i}{\mathrm{d}t}\\
    &=\frac{\mathrm{d}v_i}{\mathrm{d}t}\vec{e}_i+v_i(\epsilon_{ijk}\vec{e}_j\omega_k)\\
    &=\frac{\mathrm{d}v_i}{\mathrm{d}t}\vec{e}_i+\epsilon_{ijk}v_k\vec{e}_i\omega_j\\
    &=\qty(\frac{\mathrm{d}v_i}{\mathrm{d}t}+\epsilon_{ijk}\omega_jv_k)\vec{e}_i
  \end{split}
\end{equation}

\begin{equation}
  \frac{\mathrm{d}\vec{v}}{\mathrm{d}t}=\frac{\mathrm{d}v_i}{\mathrm{d}t}\vec{e}_i+\vec{\omega}\times\vec{v}
\end{equation}

\section{速度分解}
\begin{equation}
  \vec{r}^B=\vec{r}^A+\vec{r}^{B/A}
\end{equation}

\begin{equation}
  \begin{split}
    \vec{v}^B&=\frac{\mathrm{d}\vec{r}^B}{\mathrm{d}t}=\frac{\mathrm{d}}{\mathrm{d}t}\qty(\vec{r}^A+\vec{r}^{B/A})\\
    &=\frac{\mathrm{d}\vec{r}^A}{\mathrm{d}t}+\frac{\mathrm{d}\vec{r}^{B/A}}{\mathrm{d}t}\\
    &=\vec{v}^A+\bcancel{\frac{\mathrm{d}r^{B/A}_i}{\mathrm{d}t}\vec{e}_i}+\vec{\omega}\times\vec{r}^{B/A}\\
    &=\vec{v}^A+\vec{\omega}\times\vec{r}^{B/A}
  \end{split}
\end{equation}

\section{加速度分解}
\begin{equation}
  \begin{split}
    \vec{a}^B&=\frac{\mathrm{d}\vec{v}^B}{\mathrm{d}t}=\frac{\mathrm{d}}{\mathrm{d}t}\qty(\vec{v}^A+\vec{\omega}\times\vec{r}^{B/A})\\
    &=\frac{\mathrm{d}\vec{v}^A}{\mathrm{d}t}+\frac{\mathrm{d}}{\mathrm{d}t}\qty(\vec{\omega}\times\vec{r}^{B/A})\\
    &=\vec{a}^A+\frac{\mathrm{d}\vec{\omega}}{\mathrm{d}t}\times\vec{r}^{B/A}+\vec{\omega}\times\frac{\mathrm{d}\vec{r}^{B/A}}{\mathrm{d}t}\\
    &=\vec{a}^A+\vec{\alpha}\times\vec{r}^{B/A}+\vec{\omega}\times(\vec{\omega}\times\vec{r}^{B/A})
  \end{split}
\end{equation}

\begin{equation}
  \begin{split}
    \vec{a}^B&=\frac{\mathrm{d}\vec{v}^B}{\mathrm{d}t}=\frac{\mathrm{d}}{\mathrm{d}t}\qty(\vec{v}^A+\vec{\omega}\times\vec{r}^{B/A})\\
    &=\frac{\mathrm{d}\vec{v}^A}{\mathrm{d}t}+\frac{\mathrm{d}}{\mathrm{d}t}\qty(\vec{\omega}\times\vec{r}^{B/A})\\
    &=\vec{a}^A+\qty(\frac{\mathrm{d}}{\mathrm{d}t}(\epsilon_{ijk}\omega_jr^{B/A}_k)\vec{e}_i+\vec{\omega}\times\qty(\vec{\omega}\times\vec{r}^{B/A}))\\
    &=\vec{a}^A+\qty(\epsilon_{ijk}\alpha_jr^{B/A}_k+\bcancel{\epsilon_{ijk}\omega_j\frac{\mathrm{d}\vec{r}^{B/A}_k}{\mathrm{d}t}})\vec{e}_i+\vec{\omega}\times(\vec{\omega}\times\vec{r}^{B/A})\\
    &=\vec{a}^A+\vec{\alpha}\times\vec{r}^{B/A}+\vec{\omega}\times(\vec{\omega}\times\vec{r}^{B/A})
  \end{split}
\end{equation}

\begin{equation}
  \vec{v}^B=v^B_i\vec{e}_i=\qty(v^A_i+\epsilon_{ijk}\omega_jr^{B/A}_k)\vec{e}_i
\end{equation}

\begin{equation}
  \begin{split}
    \vec{a}^B&=\frac{\mathrm{d}\vec{v}^B}{\mathrm{d}t}=\frac{\mathrm{d}}{\mathrm{d}t}\qty(v^B_i\vec{e}_i)\\
    &=\frac{\mathrm{d}v^B_i}{\mathrm{d}t}\vec{e}_i+\vec{\omega}\times\vec{v}^B\\
    &=\qty(\frac{\mathrm{d}}{\mathrm{d}t}\qty(v^A_i+\epsilon_{ijk}\omega_jr^{B/A}_k))\vec{e}_i+\vec{\omega}\times\qty(\vec{v}^A+\vec{\omega}\times\vec{r}^{B/A})\\
    &=\qty(\frac{\mathrm{d}v^A_i}{\mathrm{d}t}\vec{e}_i+\vec{\omega}\times\vec{v}^A)+\epsilon_{ijk}\alpha_jr^{B/A}_k\vec{e}_i+\bcancel{\epsilon_{ijk}\omega_j\frac{\mathrm{d}r^{B/A}_k}{\mathrm{d}t}\vec{e}_i}+\vec{\omega}\times(\vec{\omega}\times\vec{r}^{B/A})\\
    &=\vec{a}^A+\vec{\alpha}\times\vec{r}^{B/A}+\vec{\omega}\times(\vec{\omega}\times\vec{r}^{B/A})
  \end{split}
\end{equation}

\section{角动量}
\subsection{任意点的角动量}
\begin{equation}
  \vec{H}^A_i=\vec{r}^{i/A}\times m_i\vec{v}^i
\end{equation}

\begin{equation}
  \begin{split}
    \vec{H}^A&=\int_m\vec{r}''\times\vec{v}\mathrm{d}m\\
    &=\int_m\vec{r}''\times\qty(\vec{v}^A+\vec{\omega}\times\vec{r}'')\mathrm{d}m\\
    &=\qty(\int_m\vec{r}''\mathrm{d}m)\times\vec{v}^A+\int_m\vec{r}''\times(\vec{\omega}\times\vec{r}'')\mathrm{d}m\\
    &=\vec{r}^{G/A}\times m\vec{v}^A+\int_m\vec{r}''\times(\vec{\omega}\times\vec{r}'')\mathrm{d}m
  \end{split}
\end{equation}

\subsection{固定点的角动量}
\begin{equation}
  \begin{split}
    \vec{H}^O&=\bcancel{\vec{r}^{G}\times m\vec{v}^O}+\int_m\vec{r}\times(\vec{\omega}\times\vec{r})\mathrm{d}m\\
    &=\int_m\vec{r}\times(\vec{\omega}\times\vec{r})\mathrm{d}m
  \end{split}
\end{equation}
\subsection{质心的角动量}
\begin{equation}
  \begin{split}
    \vec{H}^G&=\bcancel{\vec{r}^{G/G}\times m\vec{v}^G}+\int_m\vec{r}'\times(\vec{\omega}\times\vec{r}')\mathrm{d}m\\
    &=\int_m\vec{r}'\times(\vec{\omega}\times\vec{r}')\mathrm{d}m
  \end{split}
\end{equation}
\subsection{相互关系}
\begin{equation}
  \begin{split}
    \vec{H}^A&=\vec{r}^{G/A}\times m\vec{v}^A+\int_m\vec{r}''\times(\vec{\omega}\times\vec{r}'')\mathrm{d}m\\
    &=\vec{r}^{G/A}\times m\qty(\vec{v}^G+\vec{\omega}\times\vec{r}^{A/G})+\int_m\vec{r}''\times(\vec{\omega}\times\vec{r}'')\mathrm{d}m\\
    &=\vec{r}^{G/A}\times m\vec{v}^G+\vec{r}^{G/A}\times m(\vec{\omega}\times\vec{r}^{A/G})+\int_m(\vec{r}'+\vec{r}^{A/G})\times\qty(\vec{\omega}\times(\vec{r}'+\vec{r}^{A/G}))\mathrm{d}m\\
    &=\vec{r}^{G/A}\times m\vec{v}^G+\vec{r}^{G/A}\times m(\vec{\omega}\times\vec{r}^{A/G})+\int_m\vec{r}'\times(\vec{\omega}\times\vec{r}')\mathrm{d}m+\int_m\vec{r}'\times(\vec{\omega}\times\vec{r}^{A/G})\mathrm{d}m\\
    &+\int_m\vec{r}^{A/G}\times(\vec{\omega}\times\vec{r}')\mathrm{d}m+\int_m\vec{r}^{A/G}\times(\vec{\omega}\times\vec{r}^{A/G})\mathrm{d}m\\
    &=\vec{r}^{G/A}\times m\vec{v}^G+\vec{r}^{G/A}\times m(\vec{\omega}\times\vec{r}^{A/G})+\int_m\vec{r}'\times(\vec{\omega}\times\vec{r}')\mathrm{d}m+\bcancel{\qty(\int_m\vec{r}'\mathrm{d}m)\times(\vec{\omega}\times\vec{r}^{A/G})}\\
    &+\bcancel{\vec{r}^{A/G}\times\qty(\vec{\omega}\times\qty(\int_m\vec{r}'\mathrm{d}m))}+\vec{r}^{A/G}\times m(\vec{\omega}\times\vec{r}^{A/G})\\
    &=\vec{r}^{G/A}\times m\vec{v}^G+\int_m\vec{r}'\times(\vec{\omega}\times\vec{r}')\mathrm{d}m\\
    &=\vec{r}^{G/A}\times m\vec{v}^G+\vec{H}^G
  \end{split}
\end{equation}

\section{Inertia Tensor}
\subsection{惯性张量定义}
\begin{equation}
  \begin{split}
    \vec{H}&=\int_m\vec{r}\times(\vec{\omega}\times\vec{r})\mathrm{d}m\\
    &=\int_m((\vec{r}\cdot\vec{r})\vec{\omega}-(\vec{r}\cdot\vec{\omega})\vec{r})\mathrm{d}m\\
    &=\qty(\omega_x\int_m(x^2+y^2+z^2)\mathrm{d}m-\int_m\qty(\omega_xx+\omega_yy+\omega_zz)x\mathrm{d}m)\vec{e}_i\\
    &+\qty(\omega_y\int_m(x^2+y^2+z^2)\mathrm{d}m-\int_m\qty(\omega_xx+\omega_yy+\omega_zz)y\mathrm{d}m)\vec{e}_j\\
    &+\qty(\omega_z\int_m(x^2+y^2+z^2)\mathrm{d}m-\int_m\qty(\omega_xx+\omega_yy+\omega_zz)z\mathrm{d}m)\vec{e}_k\\
    &=\qty(\qty(\int_m(y^2+z^2)\mathrm{d}m)\omega_x-\qty(\int_mxy\mathrm{d}m)\omega_y-\qty(\int_mxz\mathrm{d}m)\omega_z)\vec{e}_i\\
    &+\qty(\qty(\int_m(z^2+x^2)\mathrm{d}m)\omega_y-\qty(\int_myz\mathrm{d}m)\omega_z-\qty(\int_myx\mathrm{d}m)\omega_x)\vec{e}_j\\
    &+\qty(\qty(\int_m(x^2+y^2)\mathrm{d}m)\omega_z-\qty(\int_mzx\mathrm{d}m)\omega_x-\qty(\int_mzy\mathrm{d}m)\omega_y)\vec{e}_k
  \end{split}
\end{equation}
\subsection{惯性张量的性质}
\begin{equation}
  \begin{split}
    \vec{H}=&
    \begin{bmatrix}
      I_{xx} & -I_{xy} & -I_{xz} \\
      -I_{yx} & I_{yy} & -I_{yz} \\
      -I_{zx} & -I_{zy} & I_{zz}
    \end{bmatrix}
    \begin{bmatrix}
      \omega_x \\
      \omega_y \\
      \omega_z
    \end{bmatrix}\\
    &=
    \begin{bmatrix}
      I_1 & 0 & 0 \\
      0 & I_2 & 0 \\
      0 & 0 & I_3
    \end{bmatrix}
    \begin{bmatrix}
      \hat{\omega}_x \\
      \hat{\omega}_y \\
      \hat{\omega}_z
    \end{bmatrix}
  \end{split}
\end{equation}
\subsection{平移轴定理}
\begin{equation}
  \begin{split}
    I^O_{xx}&=I^G_{x'x'}+m(y^2_G+z^2_G)\\
    I^O_{yy}&=I^G_{y'y'}+m(z^2_G+x^2_G)\\
    I^O_{zz}&=I^G_{z'z'}+m(x^2_G+y^2_G)\\
    I^O_{xy}&=I^G_{x'y'}+mx_Gy_G\\
    I^O_{xz}&=I^G_{x'z'}+mx_Gz_G\\
    I^O_{yz}&=I^G_{y'z'}+my_Gz_G
  \end{split}
\end{equation}

\begin{equation}
  \begin{split}
    I^O_{xx}&=\int_m(y^2+z^2)\mathrm{d}m=\int_m\qty((y'+y_G)^2+(z'+z_G)^2)\mathrm{d}m\\
    &=\int_m\qty(y'^2+z'^2)\mathrm{d}m+\int_m\qty(y_G^2+z_G^2)\mathrm{d}m+\bcancel{2y_G\int_my'\mathrm{d}m}+\bcancel{2z_G\int_mz'\mathrm{d}m}\\
    &=I^G_{x'x'}+m(y_G^2+z_G^2)
  \end{split}
\end{equation}

\begin{equation}
  \begin{split}
    I^O_{xy}&=\int_mxy\mathrm{d}m=\int_m((x'+x_G)(y'+y_G))\mathrm{d}m\\
    &=\int_mx'y'\mathrm{d}m+\int_mx_Gy_G\mathrm{d}m+\bcancel{x_G\int_my'\mathrm{d}m}+\bcancel{y_G\int_mx'\mathrm{d}m}\\
    &=I^G_{x'y'}+mx_Gy_G
  \end{split}
\end{equation} % 刚体运动学
\chapter{刚体动力学}
\section{角动量守恒定律}
\section{动力学方程}
\section{数值算法}
\section{动能}
 % 刚体动力学

\part{有限元单元理论}
\chapter{单元理论} % 单元理论
\chapter{杆单元} % 杆单元
\chapter{桁架单元} % 桁架单元
\chapter{梁单元} % 梁单元
\chapter{膜单元} % 薄膜单元
\chapter{板单元} % 板单元
\chapter{壳单元} % 壳单元
\chapter{四面体单元} % 四面体单元
\chapter{六面体单元} % 六面体单元
\chapter{实体壳单元} % 实体壳单元
\chapter{弹簧单元} % 弹簧单元
\chapter{质量单元} % 质量单元
\chapter{阻尼单元} % 阻尼单元

\end{document}
